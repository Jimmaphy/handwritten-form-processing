\section{Introduction}

% Problem / Reason
A company has a large collection of hand-written forms.
They`re considering digitizing this data.
Instead of hiring someone to manually type everything into a database,
they`re thinking about using technology to automate the process.
They`re looking for a solution that can handle different forms.
That is were the Data Science lab of the JRCZ comes in;
can an open-source cloud solution be developed that will help this costumer, and many more to come.

% Goal
The objective of this research is to explore and understand the potential of cloud computing in the interpretation,
processing, and confirmation of hand-written forms.
This involves investigating techniques for hand-writing recognition,
understanding how these techniques can be applied in a cloud computing context for repeated use,
examining the methods for processing and structuring the data,
and studying the ways to inform the user about the results.
Furthermore, a key aim is to design and develop a prototype that applies these findings,
thereby validation the results of the research.

% Research Questions
In order to achieve this goal, the following research question should be answered:
How can cloud computer be used in software to interpret, process and verify hand-written forms?
This question is split into the following sub-questions:
[1] What is hand-writing recognition?
[2] What are the techniques for interpreting hand-written forms in cloud computing?
[3] How can these techniques be applied in a cloud computing context for repeated data extraction from hand-written forms?
[4] Once data is extracted, how can it be processed and structured using cloud computing?
[5] What are the methods for confirming the results of hand-written form processing in software?
[6] How can the user be informed about the confirmation results?

% Chapter Overview
This document is split into three chapters, the first chapter being the introduction.
The second chapter will introduce the research context, focussing on cloud computing.
This chapter will explain several key components of cloud computing, referred to as `building blocks'.
A building block represents a single service offered by Microsoft Azure.
The choice of Microsoft Azure will be justified.
The potential contribution of the selected Azure services to the research will conclude the chapter.

The third chapter delves into research design.
It begin with an explanation on analyzis of the computational problem,
and how building blocks will be selected during the research.
Maintainability, a crucial aspect of any software solution, is also covered in this chapter.
The discussion on maintainability refers to the ISO 25010 standard for software quality,
ensuring that the solution is not only functional but also manageable and adaptable to future changes.